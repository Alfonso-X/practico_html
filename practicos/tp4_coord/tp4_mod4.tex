\documentclass[a4paper,12pt]{article}
\usepackage{amsmath,amsfonts,amssymb,latexsym}
\usepackage{multicol}
\usepackage{graphicx}
\usepackage[spanish]{babel}
\usepackage[utf8]{inputenc}
\usepackage[lines=44]{geometry}
\usepackage{booktabs}
\usepackage{enumerate}
\newcommand{\HRule}{\rule{\linewidth}{0.5mm}}
\newcommand{\caja}{\,\,\fbox{\hphantom{99}\vphantom{99}}\,\,}
\newcommand{\cajo}[1]{\,\,\fbox{#1\vphantom{99999}}\,\,}
\newcommand{\titolo}[1]{\Large{\bf {#1\\}}}
\newcommand{\ejemplo}[1]{\vspace{2mm}{\large{\bf{Ejemplos: }}}}
%\parindent=0pt
%\parskip .5cm
%\setlength{\oddsidemargin}{-.5in} \setlength{\evensidemargin}{-.5in}
%\setlength{\textwidth}{18.5cm}
\setlength{\textheight}{27cm}
\setlength{\footskip}{-1cm}
\setlength{\columnsep}{25mm}
%\voffset-1.3cm
\headheight-0.5cm
%\headsep 1cm
\topmargin-1.5cm
% \title{tp1_mod-4}
%\vspace{-8mm}
\begin{document}
%\thispagestyle{empty} 

\begin{center}
\Large{$\underline{\mathcal{C.E.S.P.A.\,\,N^\circ}\,\textit{4}}$}\\
\vspace{5mm}
${\mathfrak{ {Matem\acute a tica.\;\;M\acute o dulo \;4.}}}$\\
%\Large\bf{\underline{Per\'iodo de nivelaci\'on},\underline{ambientaci\'on e inte}g\underline{raci\'on}}\\
\vspace{-3mm}
\date{\today}
\end{center}
{\underline{\bf {Ciclo lectivo 2024}}}\\
\underline{\bf {Docente a car}}g\underline{\bf {o:}} Prof. Diego A. Dechat\\
\noindent
\HRule\\
 %\raggedright{Apellido y Nombre:}\hfill Curso:\hfill Fecha:\hfill.\\
%{\raggedright{Apellido y Nombre:}\hfill \hfill\hfill Agrupamiento: \hfill Fecha:\hfill.}\\
%\HRule\\

%\vspace{3mm}

{\titolo{Coordenadas Cartesianas Ortogonales en el Plano}}

\end{document}

{\Large{\bf {Fracciones equivalentes y pasaje de términos}}}\\


Hemos visto que para calcular un porcentaje tenemos que dividir el total en cien partes y multiplicar por el porcentaje pedido. Por ejemplo, para calcular el $75\%$ de $2500$, dividimos el total (que es 2500) en cien partes iguales y multiplicamos por 75. Simbólicamente, escribimos:
\begin{equation}
     75\cdot\dfrac{2500}{100}=1875   
\end{equation}
y también podemos escribir:
\begin{equation}
   \dfrac{2500}{100}=\dfrac{1875}{75} 
\end{equation}
Podemos verificar que las dos divisiones dan el mismo resultado (hacer las cuentas como actividad). En este caso decimos que las fracciones $\frac{2500}{100}$ y $\frac{1875}{75}$ son {\textbf{equivalentes}}, esto quiere decir que {\bf{representan al mismo número}}, aunque se escriben de manera diferente.\\
Notamos también que en la primer igualdad el número $75$ está multiplicando, mientras que en la segunda igualdad está dividiendo. Este cambio de lugar y de operación en un número se llama \textbf{pasaje de términos}, y se justifica mediante las \textbf{leyes de la igualdad}, una de las cuales establece que podemos multiplicar o dividir ambos miembros de una igualdad por un mismo número, sin que ella se altere (a semejanza de lo que sucede en las antiguas balanzas de platillos cuando están equilibradas: podemos sacar o agregar  la misma cantidad de cada platillo sin cambiar el estado de equilibrio). Formalmente, lo que hacemos en la primer igualdad es dividir ambos miembros por $75$, lo que mantiene la igualdad y nos permite obtener la segunda igualdad:

$$\dfrac{75}{\boldsymbol{75}}\cdot\dfrac{2500}{100}=\dfrac{1875}{\boldsymbol{75}}\text{ entonces }\boxed{\dfrac{2500}{100}=\dfrac{1875}{75}}\text{, porque } \dfrac{75}{75}=1.$$

Del mismo modo, haciendo pasaje de términos podemos tener las siguientes igualdades:

$$  \boxed{75\cdot2500=1875\cdot100},\;\boxed{\dfrac{100}{2500}=\dfrac{75}{1875}}\; \boxed{\dfrac{75}{100}=\dfrac{1875}{2500}},\; \boxed{\dfrac{2500}{1875}=\dfrac{100}{75}} $$
\newpage
{\Large{\bf {Actividades}}
\begin{enumerate}
    \item Completar los casilleros vacíos para que se verifiquen las igualdades:
    \begin{multicols}{2}
    \begin{enumerate}
    
        \item $\boxed{\vphantom{99}\hphantom{99}}\cdot7=63$
        \item $\dfrac{1}{2}=\dfrac{15}{\boxed{\vphantom{99}\hphantom{99}}}$
        \item $49\cdot\boxed{\vphantom{99}\hphantom{99}}=49$
        \item $\dfrac{\boxed{\vphantom{99}\hphantom{99}}}{36}=\dfrac{7}{6}$
        \item $\dfrac{7}{8}=\dfrac{14}{\boxed{\vphantom{99}\hphantom{99}}}$
        \item $\dfrac{2}{3}=\dfrac{8}{\boxed{\vphantom{99}\hphantom{99}}}$
         \item $\dfrac{\boxed{\vphantom{99}\hphantom{99}}}{25}=\dfrac{28}{100}$
        
    \end{enumerate}
    \end{multicols}
    \item Hallar el valor de $x$ y verificar cada resultado obtenido:
    \begin{enumerate}[a)]
         \item $10x=1800$
         \item $2x=178$
         \item $\dfrac{10}{3}x=234$
         \item $\dfrac{3}{5}x=930$
         
    \end{enumerate}
    
\end{enumerate}
\end{document}

Supogamos la siguiente situación: \\

`` El precio de costo de un pack de 6 botellas de 900 cc de aceite de girasol es de $\$2671$. ¿cuanto cuesta cada botella?''\\

Sabemos que para responder esta pregunta hay que dividir el precio del pack por seis  y vemos que la expresión decimal de la fracción $\frac{2671}{6}$ tiene muchas cifras decimales, que estas cifras decimales pueden repetirse indefinidamente y nunca tendremos resto cero.\\

{\Large{\bf{Ejemplos}}}\\
%\vspace{0.5cm}

En una calculadora, escribimos: $7\div8$; obtenemos el resultado $0,875$ (expresión decimal finita) y escribimos:
$$\frac{7}{8}=0,875\; \text{(se lee: ochocientos setenta y cinco milésimas)}.$$
Aunque en la calculadora (según el modelo) puede aparecer un punto, en nuestro país escribimos una coma para separar la parte entera de la parte decimal.

Si ahora escribimos $8\div7$, el resultado será $1,14285714286$. Según el modelo de calculadora, pueden aparecer mas o menos cifras decimales. Vemos que luego de la cifra 7, un grupo de cifras decimales comienzan a repetirse periódicamente. Este grupo de cifras que se repiten se llama {\bf período}. Este hecho es característico de los números racionales, es decir, al hacer la división entre dos números siempre encontraremos un grupo de cifras que se repite periódicamente. Como la memoria de cualquier calculadora es limitada, no puede mostrar infinitas cifras decimales, entonces la última cifra es un 6 que ``resume'' o {\bf{redondea}} las restantes cifras decimales. En este caso, escribimos
$$\frac{8}{7}=1, \widehat{142857} $$\\
%\vspace{4cm}
 (el sombrero arriba de las cifras decimales indica que este grupo de cifras se repite indefinidamente).\\
 
 Para facilitar las operaciones, podemos tomar aproximaciones {\bf por truncamiento}, que consiste en ``cortar'' el número a partir de cierta cifra decimal, y también {\bf aproximaciones por redondeo}.
 
\newpage

{\Large{\bf{Aproximación por redondeo}}}\\
%\vspace{1cm}

Cuando un número tiene muchos decimales podemos escribir una aproximación de su valor, para facilitar su escritura y las operaciones que realizamos con ellos. Dependiendo del problema a resolver y de la precisión requerida, podemos escribir un número con cualquier cantidad de decimales. El signo que usaremos para escribir el valor aproximado de un número es $\approx$, que quiere decir ``aproximadamente igual''. En nuestro caso, escribiremos los números usando solo dos decimales. 

Si un número tiene {\bf{mas}} de dos cifras decimales se puede escribir una aproximación de él con solamente dos decimales. Para ello, consideramos lo siguiente:
\begin{itemize}
\item si la tercer cifra decimal (o sea, la tercer cifra a la derecha de la coma) es {\bf{ menor}} que cinco, se escribe el número dado con solamente dos decimales, sin las otras cifras.
%\vspace{-0.5cm}

\ejemplo {} $12,4933333\approx12,49;\,\,7,574999999\approx7,57;\,\,17,78444\approx 17,78.$
\item si la  {\bf{tercer}} cifra decimal es 5 o mayor que 5, se agrega {\bf{una unidad}} a la {\bf{segunda}} cifra decimal y se eliminan las restantes. 
%\vspace{-0.5cm}

\ejemplo{} $$75,126111111\approx75,13;\,\,45,11567\approx45,12;\,\,2,71828182\approx2,72$$.

\end{itemize}
%\newpage
{\Large{\bf{Actividades}}}

\begin{enumerate}	
	
	\item Escribir la expresión decimal de los siguientes números, redondeada a dos cifras decimales:
		\begin{multicols}{3}
		\begin{enumerate}%[a)]
			\item $\dfrac{545}{\text{99}}\approx$
			\item $\dfrac{843}{999}\approx$
			\item $\dfrac{35}{99}\approx$
		      \item $\dfrac{888}{999}\approx$
			\item $\dfrac{85}{99}\approx$
			\item $\dfrac{78}{99}\approx$
			\end{enumerate}
		\end{multicols}
\item Calcular los siguientes porcentajes redondeando los resultados  a cifras enteras:
\begin{enumerate}
\begin{multicols}{3}
\item $4,13\%\text{ de }7531$
\item $37\%\text{ de }3000$
\item$40\%\text{ de }2572$
\item $7\%\text{ de }6931$
\item $22\%\text{ de }2000$
\item$40\%\text{ de }572$
\end{multicols}
\end{enumerate}
\end{enumerate}
		\end{document}
\item A partir de la información de la siguiente tabla (obtenida del sitio www.indec.gob.ar), calcular el precio actual (redondeado a dos cifras decimales)de los siguientes precios del mes pasado para la región noreste de nuestro país:

\begin{figure}[h]
\begin{center}
\includegraphics[scale=.4]{cuadro_1.png}
%\caption{Felicitación}
\end{center}
\end{figure}


		
		\end{enumerate}
		\end{document}
$$D=\{0,\,1,\,2,\,3,\,4,\,5,\,6,\,7,\,8,\,9\}$$

Además de los dígitos este sistema está constituido por órdenes, que se establecen de Derecha a Izquierda.
\newpage

% Table generated by Excel2LaTeX from sheet 'Hoja1'
\begin{table}[htbp]
  \centering
%  \caption{Add caption}
    \begin{tabular}{|c|c|c|}
   \toprule
    \hline
    3$^{er}$Orden & 2$^{do}$Orden  & 1$^{er}$Orden  \\
    %\midrule
     \hline
    centenas & decenas & unidades \\

   %\bottomrule
	\hline
      \hline
    \end{tabular}%
 % \label{tab:addlabel}%
\end{table}%



La reunión de los tres órdenes, nos establecen una Clase.
 % Table generated by Excel2LaTeX from sheet 'Hoja1'
\begin{table}[htbp]
  \centering
 % \caption{Add caption}
    \begin{tabular}{|r|r|r|r|r|r|r|r|r|r|r|r|r|r|r|r|r|r|}
 %   \toprule
\hline
\hline
    \multicolumn{3}{|c|}{6 clase} & \multicolumn{3}{c|}{5 clase} & \multicolumn{3}{c|}{4 clase} & \multicolumn{3}{c|}{3 clase} & \multicolumn{3}{c|}{2 clase } & \multicolumn{3}{c|}{1 clase } \\
  %  \midrule

\hline
    \multicolumn{3}{|p{4.5em}|}{Miles o Millares de billón} & \multicolumn{3}{p{4.36em}|}{Billón} & \multicolumn{3}{p{4.645em}|}{miles o millares de millón} & \multicolumn{3}{c|}{millón} & \multicolumn{3}{p{4.43em}|}{miles o millares} & \multicolumn{3}{p{4.43em}|}{unidades} \\
  %  \midrule

\hline
    \multicolumn{1}{|c|}{C} & \multicolumn{1}{c|}{D} & \multicolumn{1}{c|}{U} & \multicolumn{1}{c|}{C} & \multicolumn{1}{c|}{D} & \multicolumn{1}{c|}{U} & \multicolumn{1}{c|}{C} & \multicolumn{1}{c|}{D} & \multicolumn{1}{c|}{U} & \multicolumn{1}{c|}{C} & \multicolumn{1}{c|}{D} & \multicolumn{1}{c|}{U} & \multicolumn{1}{c|}{C} & \multicolumn{1}{c|}{D} & \multicolumn{1}{c|}{U} & \multicolumn{1}{c|}{C} & \multicolumn{1}{c|}{D} & \multicolumn{1}{c|}{U} \\

\hline
 %   \midrule
          &       &       &       &       &       &       &       &       &       &     4  &    2  &   0    &  2     &     5  &    0   &    0   &3\\
 %   \bottomrule

\hline

\hline
    \end{tabular}%
  \label{tab:addlabel}%
\end{table}%

Cuando se va a escribir, se anota cada dígito correspondiente a cada orden, comenzando por las superiores, y se colocan cero en el orden en que no haya dígito.
 
Por ejemplo, en la última fila de la tabla anterior escribimos el número cuarenta y dos {\bf millones} veinticinco {\bf mil} tres ubicando los dígitos en los casilleros correspondientes. Las palabras {\bf millones} y {\bf mil} están resaltadas porque nos indican a qué clase pertenecen las cifras del número que queremos formar, y completamos con ceros en los lugares que no tenemos cifras (los ceros ``hacia la izquierda'') no se escriben. Tampoco usamos comas ni puntos para separar grupos de tres cifras (en otros países los números pueden escribirse con esos separadores, y así lo muestran las calculadoras que no estén adaptadas a nuestro idioma). Podemos dejar un pequeño espacio para separar, pero no es necesario, y puede confundir. \\
  Entonces, en el sistema decimal, el número ``cuarenta y dos millones veinticinco mil tres'' se escribe: 42 025 003.\\
  El número 5000000000 se llama cinco mil millones (también, en otros países se dice ``cinco millares de millón'').\\
  
  {\titolo{Operaciones con números naturales}}
  
  Hemos dicho que utilizamos los números naturales para indicar la cantidad de elementos que tiene un conjunto determinado, por lo que si juntamos conjuntos de cosas similares tendremos un conjunto con una cantidad de elementos igual a la {\bf{suma o adición}} de las cantidades de elementos de cada conjunto original. Para obtener esta suma, se utiliza la posición relativa de cada cifra para sumarla con la cifra de la misma posición relativa en el otro número, lo que se logra sencillamente ubicando los números ``en columnas'' y pasando al siguiente orden cuando la suma de las cifras supera a 9 (estas cifras ``que se llevan'' o pasan al siguiente orden, se suelen escribir en letra de menor tamaño).\\
  Por ejemplo, para hallar la suma entre $345$ y $8761$ podemos ordenar en columnas (conviene siempre comenzar por el número mayor):
  
  \begin{table}[htbp]
  \centering
    \begin{tabular}{ccccc}
          &\scriptsize{1}&\scriptsize{1}& & \\
         &8&7&6&1\\
         +&&&&\\
         &&3&4&5\\
         \hline
         &9&1&0&6\\
         \end{tabular}
         \end{table}
Simbólicamente, se escribe $8761+345=9106$\\
Aunque no lo mencionamos antes, sabemos que el signo ``='' se lee ``es igual a'' y establece que lo que está a su izquierda es lo mismo que lo que está escrito a su derecha.\\

Cuando en una suma se repite un mismo número, decimos que tenemos una {\bf{multiplicación}}, que se representa simbólicamente usando el signo $\times$, y también un punto en la altura media de los números; por ejemplo, en lugar de escribir 
$$4+4+4 \text{ (se repite tres veces el 4)}$$
escribimos $$4\times3 \text{ , que también puede escribirse como }4\cdot3$$
Los números que intervienen en una multiplicación se llaman {\bf{factores}}, y para hallar el {\bf{producto}} (es decir, el resultado de la multiplicación) de dos números, basta con sumar uno de ellos tantas veces como lo indica el otro factor. Por ejemplo, si queremos calcular $9\times7$ y no recordamos las tablas de multiplicar, podemos resolver de la siguiente forma:
$$9\times7=\underbrace{9+9+9+9+9+9+9}_{\text{siete veces 9}}=63$$
Aunque efectiva, esta forma de multiplicar mediante sumas, es poco práctica, y conviene tener a mano las tablas de multiplicar (mejor aún, recordar {\bf{de memoria}}). Escribiremos estas tablas en la actividad número 3, completando la Tabla Pitagórica que podemos hacer sumando cada número de la primer columna consigo mismo tantas veces como indica el número correspondiente de la primer fila. Como ejemplo, en la tabla se indican algunos resultados.\\
Para obtener el producto de dos números, puede usarse un esquema en columnas similar al usado para sumar, aunque ahora las cifras que se ordenan en columnas son los resultados de multiplicar el primer número por cada cifra del segundo número, como en el siguiente\\

{\bf Ejemplo:} Hallar el siguiente resultado: 436$\cdot$23=\\
Ordenamos en columnas y multiplicamos ``de derecha a izquierda'' el primer factor por cada cifra del segundo factor, dejando un espacio vacío al cambiar de cifra (este espacio lo indicamos con un guion):
\begin{table}[htbp]
  \centering
    \begin{tabular}{cccc}
          %&\scriptsize{1}&\scriptsize{1}& & \\
         &4&3&6\\
         
         $\times$&&2&3\\
         \hline
         1&3&0&8\\
          8&7&2&-\\
           \hline
            10&0&2&8\\
         \end{tabular}
         \end{table}
         
         Las operaciones de suma y multiplicación de números naturales poseen las siguientes propiedades:
         \begin{itemize}
         \item Son {\bf cerradas} es decir, la suma y el producto de números naturales siempre es otro número natural
         \item Son operaciones {\bf conmutativas}, es decir que se puede cambiar el orden en que se suman o multiplican (el orden de los factores no altera el producto)
         \item la multiplicación es {\bf distributiva} respecto de la suma.\\
          Por ejemplo, $4\times(5+6)=4\times 5+ 4 \times 6=20+24=44$
         \end{itemize}
\newpage
{\Large\bf{ Actividades}}

\begin{enumerate}


\item Escribir los siguientes números:
	\begin{enumerate}
		\item veinticinco millones trescientos cuatro mil veinticinco.
		\item treinta y tres millones novecientos cuarenta y seis mil setecientos veintinueve.
		\item cuarenta y tres millones novecientos noventa mil cuatrocientos quince.
		\item mil quinientos treinta y seis millones cuatrocientos cincuenta y tres mil trescientos 	   			cincuenta y seis.
	\end{enumerate}
\item Escribir el nombre de los siguientes números:
	
\begin{multicols}{3}
  {
      \begin{minipage}{0.35\textwidth}
      
      a= 408 629\\
      b= 430 982 378\\
      c= 5 478 257
      \end{minipage}
    }
{
      \begin{minipage}{0.35\textwidth}
      
      d = 32 829 456\\
      e = 13 405 689\\
      f = 238 901 232
      
      \end{minipage}
    }
{
      \begin{minipage}{0.35\textwidth}
      
     g = 825 005\\
     h = 32 321 340\\
     i  = 27 890 457
         
      \end{minipage}
    }
\end{multicols}
		
\item Completar la tabla pitagórica (tabla de multiplicar):

\vspace{5mm}

\setlength{\unitlength}{1cm}
\begin{center}
\begin{picture}(10,10)
   \linethickness{0.15mm}
        \multiput(0,0)(1,0){12}{\line(0,1){11}}
      \multiput(0,0)(0,1){12}{\line(1,0){11}}
\put(0.5,10.5){\circle*{0.2}}
\put(1.4,10.4){$1$}
\put(2.4,10.4){$2$}
\put(3.4,10.4){$3$}
\put(4.4,10.4){$4$}
\put(5.4,10.4){$5$}
\put(6.4,10.4){$6$}
\put(7.4,10.4){$7$}
\put(8.4,10.4){$8$}
\put(9.4,10.4){$9$}
\put(10.4,10.4){$10$}
\put(0.4,9.4){$1$}
\put(0.4,8.4){$2$}
\put(3.4,8.4){$6$}
\put(0.4,7.4){$3$}
\put(0.4,6.4){$4$}
\put(0.4,5.4){$5$}
\put(7.4,5.4){$35$}
\put(0.4,4.4){$6$}
\put(0.4,3.4){$7$}
\put(4.4,3.4){$28$}
\put(0.4,2.4){$8$}
\put(0.4,1.4){$9$}
\put(0.4,0.4){$10$}
%	 \linethickness{0.075mm}
%	\multiput(0.5,0)(1,0){10}{\line(0,1){10}}
%        \multiput(0,0.5)(0,1){10}{\line(1,0){10}}
%		\multiput(0.25,0)(1,0){10}{\line(0,1){10}}
%        \multiput(0,0.25)(0,1){10}{\line(1,0){10}}
%	\multiput(0.75,0)(1,0){10}{\line(0,1){10}}
 %       \multiput(0,0.75)(0,1){10}{\line(1,0){10}}
%        \linethickness{0.5mm}
%	 \put(0,5){\vector(1,0){10}}
%	  \put(5,10){\vector(0,-1){10}}
%	\put(5,0){\vector(0,1){10}}
%	\put(10,5){\vector(-1,0){10}}
%%--Triángulo ABC
%	\put(6,7){\circle*{0.2}}\put(5.6,6.6){$B$}
%	\put(9,7){\circle*{0.2}}\put(9.1,6.6){$A$}
%	\put(9,9){\circle*{0.2}}\put(8.7,9.2){$C$}
%	{\linethickness{0.5mm}
%	\put(6,7){\line(1,0){3}}%--Segmento BA
%	\put(9,7){\line(0,1){2}}}%--Segmento AC	}	
       %\linethickness{2.5mm} 
%       \put(6,7){\line(3,2){3.1}}%--Segmento BC
\end{picture}    
\end{center}
\item Escribir (en números y letras) las siguientes operaciones y sus resultados:
	\begin{enumerate}
		\item El triple de setenta y cinco mil:
		\item El doble de novecientos cincuenta y dos:
		\item El cuádruple  de cuarenta y siete mil ochocientos veintitrés:
	\end{enumerate}


\item Ordenar en columnas y resolver:
\begin{enumerate}[a)]
\begin{multicols}{3}
  {
   %   \begin{minipage}{0.5\textwidth}
      
   \item   14301+10+2535+403=
   \item   10459+23409+25+1501=
      
     % \end{minipage}
    }
{
 %     \begin{minipage}{0.25\textwidth}
      
    \item  14012-12897=\\
    \item  25673-2345=
      
     % \end{minipage}
    }
{
  %    \begin{minipage}{0.25\textwidth}
      
   \item  $ 26507\cdot27$=\\
 \item    $ 4578\cdot407$=
      
    %  \end{minipage}
    }
\end{multicols}
\end{enumerate}
\end{enumerate}
\end {document}
\item Resolver y responder adecuadamente los siguientes problemas:

\begin{enumerate}
	\item Un producto cuesta \$379,99. ¿Qué operación hay que realizar para saber cuanto 				costarán tres productos iguales?
	\item El precio de una caja con 12 botellas de aceite es de \$690. ¿Qué operación podemos 				realizar para conocer el precio de cada botella?¿cuanto costarán dos botellas de aceite?
	\item Una persona camina 75 m por minuto. Expresar, primero en metros y luego en 					kil\'ometros,   la distancia que recorre en una hora.
	\item Un terreno rectangular mide 8 m de frente por 24 m de fondo, y las medidas de otro 	 	  	terreno con la misma forma son 10 m de frente por 18 m de fondo. ¿Cual tiene mayor 	 	 	superficie? 
	\item En un comercio se ofrece un paquete de 8 pañales descartables a \$90 ¿cuanto 					cuesta  cada pañal? ¿cuanto saldría 40 pañales de la misma marca y tamaño? 
\end{enumerate}
\end{enumerate}

\end{document}
	\item Obtener el resultado de las siguientes operaciones:

\item Otra marca de pañal del mismo tamaño que el anterior se vende a \$150 el paquete de 40  	unidades. Obtener el costo de cada pañal y comparar con el valor obtenido en el ejercicio 	 	anterior. ¿Cual resulta más barato por unidad?
\item Si 12 cajas tetra brik de jugo cuestan \$174,¿cuanto cuestan 15 cajas del mismo producto?
\item El costo de 1 Kg de pan es de \$18, ¿que cantidad podemos comprar por \$10?
\item El kilo de uva rosada vale \$31,49. ¿Qué cantidad podemos comprar por \$50?

\item La siguiente informaci\'on, referida a los precios de la canasta escolar en la provincia de Corrientes, fue extra\'ida del diario ``\'Epoca'' del 17 de febrero de 2017:

\begin{tabular*}{\textwidth}{@{\extracolsep{\fill}} | l  r | l  r | }
\hline
\multicolumn{4}{|c|}{Productos incluidos} \\ \hline
  \hline
Repuesto N$^\circ$3 x 96 hojas   & \$ 21.75 &Repuesto Cans\'on N$^\circ$ 5 x 8 hojas &\$ 8.33   \\
  \hline 
 Cuaderno tipo espiral 80 hojas &\$ 31.59    & Cuaderno rayado tapa flex. 48 hojas& \$ 8.70    \\
  \hline
 Cuaderno x 82 hojas, ara\~na& \$ 30.45& Carpeta N$^\circ$ 5 fibra negra &\$ 19.14\\
\hline
 Carpeta 3 anillos 40 mm fibra negra& \$ 37.41& L\'apiz Negro& \$ 1.74 \\
\hline
Bol\'igrafo &\$ 4.35& Borrador Tinta L\'apiz &\$ 2.61 \\
\hline
Borrador L\'apiz &\$ 2.61& Set de Geometr\'ia &\$ 18.27 \\
\hline
Regla 20 cm &\$ 4.35& Comp\'as &\$ 10.44 \\
\hline
Cartuchera &\$ 22.62& Sacapuntas &\$ 1.74\\
\hline
 Tijerita& \$ 12.18& Crayones x 12 uni.& \$ 21.75 \\
\hline
Pincel N$^\circ$ 4&\$ 3.25& T\'emperas color Surtido& \$ 41.76\\
\hline
Marcadores Fibras x 10 uni.& \$ 24.36 & Marcador grueso punta redonda &\$ 6.09 \\
\hline
 L\'apices de colores x 12 uni. Larg. &\$ 23.49 & L\'apiz corrector &\$ 11.31 \\
\hline
Adhesivo Vin\'ilico &\$ 3.48&Goma Eva &\$ 12.18\\
\hline
Papel glas\'e &\$ 1.74& Papel Crepe &\$ 6.09\\
\hline
 Cartulina &\$ 4.35& Mapa Arg. / Planisferio/Ctes. &\$ 0.87\\
\hline
subtotal& &subtotal &\\
\hline
\multicolumn{4}{|l|}{TOTAL} \\ \hline
\hline

\end{tabular*}


Comprando en efectivo, el kit implica un descuento del 32\% con relaci\'on a los precios de g\'ondola. Utilizando la tarjeta de d\'ebito del Banco de Corrientes, suma un descuento adicional de 30\%; optando por la tarjeta de cr\'edito de la entidad, el beneficio es de 20\% m\'as doce cuotas sin inter\'es. Tambi\'en se pueden comprar por separado de acuerdo a la necesidad de cada familia. Ya se encuentran disponibles en Comercios de Capital y en nueve localidades hasta el 15 de marzo de 2017.

{\bf\large{Actividades}}

\begin{enumerate}
	
	\item ¿Cu\'antos art\'iculos hay en la tabla? ¿Cuales son los artículos de mayor y de menor 	 		costo, respectivamente?
	\item Obtener el costo total del kit (sumar cada columna de precios y luego hallar la suma 	 		total).
	\item Obtener el costo si solamente compramos los siguientes productos: 1 Repuesto                           		N$^\circ$3 x 96 hojas, 1 carpeta 3 anillos 40 mm fibra negra, 1 cuaderno tipo espiral de 	 	80 hojas, 1 bol\'igrafo, 1 borrador para l\'apiz, 1 l\'apiz negro, un set de geometr\'ia, 1 	     	 		comp\'as y 1 l\'apiz corrector.
	
      	 \item Elaborar una lista de art\'iculos con sus precios respectivos para un ni\~no en etapa 		               inicial (jard\'in de infantes) y obtener el costo total.
	\item Calcular el dinero descontado por pago con  tarjeta de d\'ebito. (multiplicar el importe 	  		total por el porcentaje de descuento y dividir por 100)
	\item Obtener el importe de cada cuota si se elige la opci\'on de pago con tarjeta de cr\'edito 		y en doce cuotas.
	

\end{enumerate}

\vspace{-3mm}
\end{enumerate}
\vspace{-3mm}
\end{document}%%%%%%%%%%%%%%%%%%%%%%%%%%%%%%%%%


El siguiente trabajo debe realizarse en forma individual o en grupos de dos o tres integrantes. Responda cada pregunta de la forma mas clara y prolija posible. Trate de resolver todas las operaciones en una hoja de carpeta, detallando los pasos realizados.
\vspace{-3mm}
\begin{enumerate}
	\item Averiguar cuanto tiempo (en horas y minutos) hay:\\
		\begin{enumerate}
			\item entre las 13:00 y las 20:45
			\item desde las 8:00 hasta las 17:30 
			\item desde las 19:15 hasta las 23:00.
		\end{enumerate}
\item Una persona camina 75 m por minuto. Expresar, primero en metros y luego en kil\'ometros, la distancia que recorre en una hora.

\item En un año bisiesto como el 2016, al calendario se le agrega un día extra, que es el 29 de febrero. 
		\begin{enumerate}
                   \item  ¿Cuanto tiempo pasó desde el último año bisiesto?¿cuando será el próximo?
			\item  ¿Cuantos días tiene un año bisiesto?
			\item ¿Cu\'antos años bisiestos hubo desde 1990 hasta hoy?
		\end{enumerate}

\item En un comercio se ofrece un paquete de 8 pañales a \$35 ¿cuanto cuesta cada pañal? ¿cuanto saldría 40 pañales de la misma marca y tamaño?

\item Si 12 cajas de jugo cuestan \$174,¿cuanto cuestan 2 cajas del mmismo jugo?

\end{enumerate}
\end{document}
\item Las notas obtenidas por los 36 alumnos en un curso de Matemáticas son las siguientes:
$$6,\;1,\;10,\;2,\;3,\;4,\;7,\;7,\;2,\;3,\;6,\;8,\;8,\;9,\;10,\;4,\;4,\;2,\;8,\;8,\;3,\;9,\;7,\;10,\;1,\;1,\;6,\;5,\;5,\;8,\;9,\;9,\;6,\;8,\;7,\;4.$$
\begin{enumerate}
\item Elaborar la tabla de frecuencias de notas.
\item Obtener  el porcentaje de alumnos aprobados (Nota $\geq$6) y desaprobados. 
\item Representar mediante sectores circulares los porcentajes de aprobados y desaprobados.
\end{enumerate}
\item Representar mediante sectores circulares los siguientes porcentajes:
\begin{enumerate}
\begin{multicols}{3}
\item 32\%,\;10\%,\;58\%.
\item 25\%,\;15\%,\;20\%,40\%.
\item 57\%,\;23\%,\;15\%,\;5\%.
\end{multicols}
\end{enumerate}
\item Resolver las siguientes ecuaciones y verificar los resultados obtenidos:
\begin{enumerate}
\begin{multicols}{3}
\item $2x+13=-1$
\item $\displaystyle\frac{2}{3}x-\frac{4}{5}=8$
\item $-\displaystyle\frac{2}{5}x-\frac{3}{7}=\frac{3}{5}x-\frac{5}{2}$
\item $4x-33=-20$
\item $\displaystyle\frac{5}{4}x-\frac{1}{6}=18$
\item $-\displaystyle\frac{3}{2}x-\frac{3}{7}=\frac{3}{2}x-\frac{2}{5}$
\item $x+\frac{1}{3}x=90$
\item $x- \displaystyle\frac{4}{100}x=168$
\item $12+\displaystyle\frac{12}{100}x=14$
\end{multicols}
\end{enumerate}
\item Para cada uno de los siguientes problemas, plantear la ecuación correspondiente, resolverla, verificar el resultado obtenido y responder adecuadamente las preguntas:
\begin{enumerate}
\item Dos ángulos son complementarios y uno de ellos es la tercera parte del otro.¿Cuanto mide cada ángulo?
\item ¿De qué número es 265 el 6\% mas?
\item ¿De qué número es 168 el 4\% menos?
\item Un producto cuyo precio era de \$11 hoy se vende a \$ 13.¿Cual es el porcentaje de aumento?
\item El precio de cierto artículo era \$590. En una promoción se lo vende a \$320. ¿cual fue el porcentaje descontado?
\end{enumerate}
\end{enumerate}


\end{document}
